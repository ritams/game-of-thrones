\chapter{Digging into the Data: The Foundation of Electoral Analysis}
\label{chap3}

In the previous chapter, we explored how random nudges can effectively reduce polarization in online social networks. We now shift our focus to a different but related arena of collective decision-making: democratic elections. Just as individual opinions can aggregate into social patterns, individual votes aggregate into electoral outcomes. Elections represent the ultimate test of democratic opinion dynamics at scale, where millions of citizens express their preferences through a structured process. However, before we can uncover any meaningful patterns or universals in electoral behavior, we must first establish a solid empirical foundation through careful data collection, preparation, and analysis.

\section{The Pivot: From Opinions to Elections}

Individual opinions ultimately aggregate into collective choices through elections. While the previous chapter dealt with the dynamics of opinion formation and the potential interventions to mitigate polarization, this chapter focuses on the empirical foundation of electoral analysis. Elections represent the most structured and widespread manifestation of collective decision-making, offering a rich dataset for studying how individual preferences translate into societal outcomes.

The quality of data is paramount for any meaningful electoral analysis. Without robust, comprehensive, and well-curated data, theoretical insights remain untethered from reality. This chapter details our extensive data collection and preparation efforts, establishing the empirical backbone that will enable the discovery of universal patterns in subsequent chapters.

\section{The Great Data Hunt: Scraping Democracy's Digital Footprints}

To conduct a comprehensive analysis of electoral patterns across different democratic systems, we compiled election data from 34 countries spanning six continents. Our data sources included:

\begin{itemize}
    \item Constituency-Level Election Archive (CLEA) \cite{KliYegHan2012}
    \item National election commission websites from various countries \cite{BorRayBou2012}
    \item MIT Election Data and Science Lab \cite{FerSucRam2014}
    \item Secondary sources for historical elections \cite{ChaMitFor2013}
\end{itemize}

The data collection process faced several technical challenges, particularly due to the inconsistent formats in which electoral data is published worldwide. We employed semi-automated scraping techniques using Python libraries to extract data from sources ranging from structured databases to machine-generated PDFs. For each country, we aimed to collect data from multiple elections spanning several years to ensure temporal robustness.

For countries like India, we collected data from multiple electoral scales, including polling booth level (approximately $10^2$ voters), assembly constituency level (approximately $10^5$ voters), and parliamentary constituency level (approximately $10^6$ voters) \cite{BokSzaVat2018}. This multi-scale approach allows us to investigate scale-dependent electoral phenomena that have rarely been studied systematically.

\section{A Peek Behind the Curtain: Sample Dataset Structure}

The raw election data collected across various countries contained several key variables:
\begin{itemize}
    \item Voter turnout ($T$): The proportion of registered voters who cast ballots
    \item Candidate vote shares: The proportion or number of votes received by each candidate
    \item Winning margin ($M$): The difference between the vote shares of the winner and the runner-up
    \item Constituency identifiers: Geographic or administrative codes for electoral units
    \item Temporal information: Election dates and cycles
\end{itemize}

The scale of our dataset presents both challenges and opportunities. By including data from electoral units of vastly different sizes—from small polling booths with around 100 voters to large parliamentary constituencies with over a million voters—we can investigate how electoral statistics scale with the number of voters. This multi-scale analysis is particularly prominent in our Indian election data, where we have consistent information across three distinct electoral levels \cite{mantovani2011scaling}.

\section{Data Cleaning: The Unglamorous but Critical Foundation}

The raw data required extensive cleaning and standardization before it could be used for statistical analysis. We implemented a systematic approach to handle common issues:

\begin{itemize}
    \item Missing values were addressed through appropriate imputation techniques or, when necessary, by excluding affected data points
    \item Inconsistent formats were standardized, particularly for turnout and margin calculations
    \item Encoding issues, especially for non-Latin script languages, were resolved through Unicode normalization
\end{itemize}

To ensure statistical robustness, we applied several filtering criteria:
\begin{itemize}
    \item Minimum threshold of 400 data points per country to ensure statistical significance \cite{BorRayBou2012}
    \item Exclusion of data points with zero turnouts or single-candidate races
    \item Requirement of at least two viable candidates per constituency
\end{itemize}

These criteria reduced our initial pool of potential countries from approximately 180 to the final 34 included in our analysis. For multi-round electoral systems, we focused primarily on the final decisive round, though we maintained data from preliminary rounds for supplementary analyses.

A particularly challenging aspect of longitudinal electoral data is boundary redistricting, where constituency boundaries change over time. Where possible, we tracked these changes and created consistent geographic units for temporal analysis. When this was not feasible, we treated elections before and after redistricting as separate statistical ensembles \cite{BraDeA2017}.

\section{The Numbers Speak: Key Statistics and Distributions}

Our cleaned and standardized dataset reveals striking patterns and variations in electoral statistics across countries. Key summary statistics include:
\begin{itemize}
    \item Mean turnout rates ranging from approximately 45\% to 90\% across different democratic systems \cite{BorRayBou2012}
    \item Mean margins of victory showing significant variation between established and emerging democracies
    \item Standard deviations revealing the degree of electoral competitiveness within each country
\end{itemize}

Table \ref{tab:election_stats} presents summary statistics for our dataset, highlighting the diversity of electoral scales and patterns across different democratic systems.

\begin{table}[h]
\centering
\caption{Summary Statistics of Election Data}
\label{tab:election_stats}
\begin{tabular}{|c|c|c|c|c|c|c|c|}
\hline
Country & Time span & Number & Scale    & Mean turnout & Mean margin  & Number \\ 
 &  & of  &  &  &  & of electoral \\ 
 &  & elections  &  &  &  & units \\ 
 &  &  &  &  &  & (consolidated) \\ \hline
Australia & 1901-2016 & 37 & Constituency & $7.37\times 10^{4}$ & $1.31\times 10^{4}$ & 1740\\ \hline
Bangladesh & 1973-2008 & 4 & Constituency & $1.57\times 10^{5}$ & $3.15\times 10^{4}$ & 1188\\ \hline
Belarus & 2004-2019 & 5 & Constituency & $4.83\times 10^{4}$ & $2.61\times 10^{4}$ & 441\\ \hline
Canada & 1867-2019 & 43 & Constituency & $2.76\times 10^{4}$ & $5.50\times 10^{3}$ & 10662\\ \hline
Canada & 2004-2021 & 7 & Polling Booth & $5.56\times 10^{2}$ & $1.35\times 10^{2}$ & 489919\\ \hline
Chile & 1945-2017 & 7 & Constituency & $1.07\times 10^{5}$ & $1.05\times 10^{4}$ & 420\\ \hline
Denmark & 1849-2019 & 30 & Constituency & $2.70\times 10^{3}$ & $4.64\times 10^{2}$ & 2178\\ \hline
Ethiopia & 2010-2010 & 1 & Constituency & $4.95\times 10^{4}$ & $4.18\times 10^{4}$ & 492\\ \hline
France & 1973-2017 & 3 & Constituency & $7.88\times 10^{4}$ & $1.10\times 10^{4}$ & 1712\\ \hline
Germany & 1871-2017 & 19 & Constituency & $1.37\times 10^{5}$ & $2.26\times 10^{4}$ & 5108\\ \hline
Ghana & 1992-2016 & 6 & Constituency & $3.75\times 10^{4}$ & $9.88\times 10^{3}$ & 1410\\ \hline
Hungary & 1990-2018 & 6 & Constituency & $5.32\times 10^{4}$ & $8.57\times 10^{3}$ & 936\\ \hline
India & 1951-2019 & 18 & Constituency & $5.69\times 10^{5}$ & $8.33\times 10^{4}$ & 8389\\ \hline
India & 2004-2019 & 4 & Polling Booth & $5.82\times 10^{2}$ & $1.89\times 10^{2}$ & 752786\\ \hline
Japan & 1947-2017 & 26 & Constituency & $2.88\times 10^{5}$ & $2.35\times 10^{4}$ & 4603\\ \hline
Kenya & 1961-2013 & 2 & Constituency & $3.72\times 10^{4}$ & $1.19\times 10^{4}$ & 417\\ \hline
Korea & 1948-2012 & 13 & Constituency & $6.17\times 10^{4}$ & $1.01\times 10^{4}$ & 2258\\ \hline
Lithuania & 1992-2020 & 8 & Constituency & $3.24\times 10^{4}$ & $3.98\times 10^{3}$ & 570\\ \hline
Malawi & 1994-2019 & 4 & Constituency & $2.31\times 10^{4}$ & $6.29\times 10^{3}$ & 755\\ \hline
Malaysia & 1959-2018 & 13 & Constituency & $3.41\times 10^{4}$ & $8.90\times 10^{3}$ & 2199\\ \hline
Myanmar & 2010-2015 & 2 & Constituency & $6.76\times 10^{4}$ & $2.32\times 10^{4}$ & 634\\ \hline
New Zealand & 1943-2020 & 9 & Constituency & $3.04\times 10^{4}$ & $6.94\times 10^{3}$ & 637\\ \hline
Nigeria & 2003-2019 & 2 & Constituency & $7.75\times 10^{4}$ & $2.20\times 10^{4}$ & 710\\ \hline
Pakistan & 1988-2013 & 3 & Constituency & $1.28\times 10^{5}$ & $2.45\times 10^{4}$ & 683\\ \hline
Papua New Guinea & 1972-2017 & 8 & Constituency & $5.07\times 10^{4}$ & $5.66\times 10^{3}$ & 841\\ \hline
Philippines & 1946-2013 & 17 & Constituency & $1.83\times 10^{5}$ & $2.63\times 10^{4}$ & 2525\\ \hline
Solomon Islands & 1967-2019 & 14 & Constituency & $3.67\times 10^{3}$ & $4.37\times 10^{2}$ & 543\\ \hline
Taiwan & 1986-2020 & 11 & Constituency & $2.33\times 10^{5}$ & $1.98\times 10^{4}$ & 482\\ \hline
Tanzania & 2005-2020 & 2 & Constituency & $5.37\times 10^{4}$ & $2.01\times 10^{4}$ & 492\\ \hline
Thailand & 1969-2011 & 12 & Constituency & $1.86\times 10^{5}$ & $1.46\times 10^{4}$ & 2263\\ \hline
Trinidad and Tobago & 1925-2020 & 13 & Constituency & $1.53\times 10^{4}$ & $5.12\times 10^{3}$ & 411\\ \hline
Uganda & 2006-2021 & 4 & Constituency & $4.45\times 10^{4}$ & $1.08\times 10^{4}$ & 1430\\ \hline
UK & 1832-2019 & 46 & Constituency & $3.43\times 10^{4}$ & $6.30\times 10^{3}$ & 23105\\ \hline
Ukraine & 1998-2019 & 5 & Constituency & $8.89\times 10^{4}$ & $1.67\times 10^{4}$ & 1072\\ \hline
United States & 1788-2020 & 167 & Congressional District & $1.14\times 10^{5}$ & $2.96\times 10^{4}$ & 33946\\ \hline
United States & 2000-2020 & 6 & County & $1.78\times 10^{5}$ & $2.00\times 10^{4}$ & 18905\\ \hline
Zimbabwe & 2005-2018 & 4 & Constituency & $1.77\times 10^{4}$ & $6.55\times 10^{3}$ & 743\\ \hline
\end{tabular}
\end{table}

The table illustrates several important patterns. First, there is considerable variation in the scale of electoral units across countries, with mean turnout ranging from hundreds of voters in Canadian polling booths to hundreds of thousands in Indian constituencies. Second, the ratio of mean margin to mean turnout—a measure of electoral competitiveness—varies significantly across democratic systems. Third, the dataset spans extensive temporal periods, with some countries represented by over a century of electoral data.

\section{Data as the Single Most Important Foundation}

This extensive dataset forms the empirical backbone of our subsequent analyses. Without this robust, comprehensive collection of electoral data, theoretical insights would remain untethered from reality. The quality of our data curation—emphasizing comprehensiveness, accuracy, and cross-system comparability—enables us to pursue the discovery of universal patterns in democratic elections.

Quality has been prioritized over mere quantity in our approach. While expanding to more countries might seem advantageous, we have focused on ensuring that each included country has sufficient data points and meets our quality thresholds. This careful curation provides a solid foundation for the statistical analyses and theoretical modeling in subsequent chapters.

The stage is now set for our global hunt for electoral universals. With this meticulously prepared dataset spanning diverse geographical, cultural, and institutional contexts, we can begin to search for patterns that transcend the particularities of individual democratic systems. The next chapter will leverage this empirical foundation to uncover surprising universality in electoral competition across vastly different democratic contexts.
