\chapter{Digging into the Data: The Foundation of Electoral Analysis}
\label{chap3}

In the previous chapter, we explored how random nudges can effectively reduce polarization in online social networks. We now shift our focus to a different but related arena of collective decision-making: democratic elections. Just as individual opinions can aggregate into social patterns, individual votes aggregate into electoral outcomes. Elections represent the ultimate test of democratic opinion dynamics at scale, where millions of citizens express their preferences through a structured process. However, before we can uncover any meaningful patterns or universals in electoral behavior, we must first establish a solid empirical foundation through careful data collection, preparation, and analysis.

\section{The Pivot: From Opinions to Elections}

Individual opinions ultimately aggregate into collective choices through elections. While the previous chapter dealt with the dynamics of opinion formation and the potential interventions to mitigate polarization, this chapter focuses on the empirical foundation of electoral analysis. Elections represent the most structured and widespread manifestation of collective decision-making, offering a rich dataset for studying how individual preferences translate into societal outcomes.

The quality of data is paramount for any meaningful electoral analysis. Without robust, comprehensive, and well-curated data, theoretical insights remain untethered from reality. This chapter details our extensive data collection and preparation efforts, establishing the empirical backbone that will enable the discovery of universal patterns in subsequent chapters.

\section{Data Collection Methodology and Sources}

To conduct a comprehensive analysis of electoral patterns across different democratic systems, we compiled election data from 34 countries spanning six continents. Our data collection began with an ambitious survey of 180 countries and territories worldwide, from which we ultimately selected 34 countries that met our stringent quality criteria for statistical analysis.

Our primary data sources included the Constituency-Level Election Archive (CLEA) website for constituency-level data of lower chamber legislative elections across the globe \cite{clea}. For specialized datasets, we collected polling booth level data for India and Canada directly from their respective Election Commission websites \cite{india_data, canada_data}, employing semi-automated techniques using Python libraries to handle the diverse data formats. County-level data for the United States was obtained from the MIT Election Data and Science Lab \cite{DVN/VOQCHQ_2018}.

The data collection process faced considerable technical challenges due to the heterogeneous formats in which electoral data is published worldwide. While constituency-level data from CLEA was consistently formatted in tabular structures, polling booth-level data presented significant extraction challenges, ranging from structured tabular formats to machine-generated and scanned PDFs. We developed robust data cleaning procedures using combinations of Python libraries to handle these varied formats systematically.

For countries like India, we successfully collected data across multiple electoral scales that span three orders of magnitude in voter numbers. At the finest granularity, polling booth level data encompasses approximately $10^2$ voters per unit, assembly constituency level data covers approximately $10^5$ voters, and parliamentary constituency level extends to approximately $10^6$ voters. This multi-scale approach enables investigation of scale-dependent electoral phenomena that have rarely been studied systematically across such a broad range of democratic contexts.

\section{Dataset Structure}

The raw election data collected across various countries contained several essential variables that form the foundation of our electoral analysis. Voter turnout represents the actual number of voters who cast ballots in each electoral unit, serving as a crucial scale parameter for our statistical models. Candidate vote tallies provide the detailed breakdown of votes received by each competing candidate, from which we derive winner and runner-up vote counts. The winning margin, defined as the difference between votes secured by the winner and runner-up, captures the competitiveness of each electoral contest. Geographic and administrative identifiers enable spatial analysis and constituency-level comparisons, while temporal information spanning election dates and cycles allows for longitudinal analysis across multiple electoral cycles.

The scale hierarchy in our dataset presents unprecedented opportunities for multi-scale electoral analysis. Indian election data exemplifies this hierarchical structure most comprehensively, encompassing polling booth level data with typical turnouts around 583 voters, assembly constituency level with approximately 116,577 voters for general elections and 86,484 voters for state elections, and parliamentary constituency level reaching 587,329 voters on average. This three-orders-of-magnitude variation in electoral unit sizes enables systematic investigation of how electoral statistics scale with voter population, a phenomenon that has received limited systematic study in the electoral systems literature.
\begin{figure}[H]
    \centering
    \includegraphics[width=\linewidth]{chapters/chapter3/clea.png}
    \caption{A representative example of the constituency-level election data from CLEA.}
    \label{fig:clea}
\end{figure}

\begin{figure}[H]
    \centering
    \includegraphics[width=\linewidth]{chapters/chapter3/eci_pc.png}
    \caption{A representative example of the constituency-level election data from Election Commission of India.}
    \label{fig:eci_pc}
\end{figure}

\begin{figure}[H]
    \centering
    \includegraphics[width=\linewidth]{chapters/chapter3/form_20.png}
    \caption{A representative example of the form 20 which contains the polling booth-level election data from Election Commission of India.}
    \label{fig:form_20}
\end{figure}
Figs \ref{fig:clea}, \ref{fig:eci_pc}, and \ref{fig:form_20} show representative examples of the constituency-level, parliamentary constituency-level, and polling booth-level election data from CLEA, Election Commission of India, and Election Commission of India respectively. While the CLEA data is in a tabular format, the data from Election Commission of India is in PDF format, which poses a challenge for automated data extraction.
\section{Data Processing and Quality Assurance}
The raw data required extensive cleaning and standardization procedures to ensure analytical integrity. Our systematic data cleaning approach addressed multiple technical challenges inherent in cross-national electoral data compilation. Missing values were handled through careful exclusion of affected data points rather than imputation, preserving the authenticity of electoral outcomes. Format inconsistencies were systematically standardized, with particular attention to turnout and margin calculations to ensure cross-country comparability. Encoding issues, particularly prevalent in data from countries using non-Latin script languages, were resolved through comprehensive Unicode normalization procedures.

Statistical robustness requirements drove our implementation of stringent filtering criteria that ultimately shaped our final dataset. We established a minimum threshold of 400 consolidated data points per country to ensure adequate statistical power for universality demonstration while maintaining capability for electoral fraud detection. This threshold proved effective for identifying potential electoral misconduct in Ethiopia and Belarus while preserving robust statistical foundations for analysis. Cases with zero turnout were systematically excluded from analysis, as were electoral units with fewer than two competing candidates, ensuring that our margin calculations reflected genuine electoral competition.

Our filtering criteria necessitated significant dataset reduction from the initial survey of 180 countries to the final analytical set of 34 countries. This reduction prioritized data quality over mere quantity, ensuring that each included country contributed statistically meaningful information to our cross-national analysis. To maintain consistency in our turnout calculations, we defined turnout as the sum of valid votes received by all candidates in each electoral unit, avoiding discrepancies that might arise from different approaches to calculating voter participation rates.

The temporal consolidation of multiple elections from each country created unified statistical ensembles that capture long-term electoral patterns while maintaining sufficient sample sizes for robust statistical inference. For Indian elections specifically, our dataset spans from 1962 to 2019 for parliamentary constituencies, 1999 to 2019 for assembly constituencies in general elections, 2004 to 2019 for polling booth level data, and 1961 to 2023 for state assembly elections, providing comprehensive temporal coverage across multiple electoral scales and types.

\section{Descriptive Statistics and Cross-National Patterns}

Our cleaned and standardized dataset reveals remarkable diversity in electoral patterns across countries and scales, providing insights into the fundamental characteristics of democratic competition worldwide. The consolidated electoral data demonstrates substantial variation in mean turnout values, ranging from Denmark's constituency-level average of approximately 2,700 voters to India's parliamentary constituency average exceeding 587,000 voters. Similarly, mean margins of victory exhibit considerable cross-national variation, reflecting different degrees of electoral competitiveness across democratic systems.

The temporal depth of our dataset provides exceptional analytical power, with some countries contributing over 150 years of electoral history. The United States congressional district data spans from 1788 to 2020 across 167 elections, while the United Kingdom's constituency data covers 1832 to 2019 through 46 electoral cycles. This extensive temporal coverage enables investigation of long-term electoral trends and system evolution across different democratic contexts.

Table \ref{tab:election_stats} presents comprehensive summary statistics for our dataset, illustrating the extraordinary diversity of electoral scales and patterns across different democratic systems. The table demonstrates the multi-scale nature of our analysis, with electoral unit sizes varying by more than three orders of magnitude within single countries and by over four orders of magnitude across the complete dataset.

The summary statistics reveal several fundamental patterns in our comprehensive dataset. Electoral unit scales demonstrate extraordinary variation both within and across countries, exemplified by the contrast between Canadian polling booths averaging 556 voters and Indian parliamentary constituencies averaging 587,329 voters. The consolidated number of electoral units provides insight into the statistical power available for each country, ranging from Ethiopia's limited 492 units to India's extensive polling booth dataset encompassing 752,786 units.

Cross-national analysis reveals significant differences in electoral competitiveness as measured by the margin-to-turnout ratio. Countries like Denmark and Solomon Islands exhibit relatively high competitiveness with smaller margins relative to turnout, while others show broader victory margins suggesting different competitive dynamics. These variations reflect the complex interplay of electoral systems, political culture, and institutional arrangements across different democratic contexts.

The temporal coverage demonstrates the robustness of our longitudinal analysis, with consolidated data representing decades or centuries of democratic evolution. Countries like Australia contribute 37 elections from 1901 to 2016, while Canada's constituency-level data spans an remarkable 43 elections from 1867 to 2019. This extensive temporal depth enables investigation of how electoral patterns evolve over time within stable democratic systems, providing insights into the persistence or change of competitive structures across different historical periods.
\newpage
\begin{table}[H]
\centering
{\scriptsize
\begin{tabular}{|c|c|c|c|c|c|c|c|}
\hline
Country & Time span & Number & Scale    & Mean turnout & Mean margin  & Number \\ 
 &  & of  &  &  &  & of electoral \\ 
 &  & elections  &  &  &  & units \\ 
 &  &  &  &  &  & (consolidated) \\ \hline
Australia & 1901-2016 & 37 & Constituency & $7.37\times 10^{4}$ & $1.31\times 10^{4}$ & 1740\\ \hline
Bangladesh & 1973-2008 & 4 & Constituency & $1.57\times 10^{5}$ & $3.15\times 10^{4}$ & 1188\\ \hline
Belarus & 2004-2019 & 5 & Constituency & $4.83\times 10^{4}$ & $2.61\times 10^{4}$ & 441\\ \hline
Canada & 1867-2019 & 43 & Constituency & $2.76\times 10^{4}$ & $5.50\times 10^{3}$ & 10662\\ \hline
Canada & 2004-2021 & 7 & Polling Booth & $5.56\times 10^{2}$ & $1.35\times 10^{2}$ & 489919\\ \hline
Chile & 1945-2017 & 7 & Constituency & $1.07\times 10^{5}$ & $1.05\times 10^{4}$ & 420\\ \hline
Denmark & 1849-2019 & 30 & Constituency & $2.70\times 10^{3}$ & $4.64\times 10^{2}$ & 2178\\ \hline
Ethiopia & 2010-2010 & 1 & Constituency & $4.95\times 10^{4}$ & $4.18\times 10^{4}$ & 492\\ \hline
France & 1973-2017 & 3 & Constituency & $7.88\times 10^{4}$ & $1.10\times 10^{4}$ & 1712\\ \hline
Germany & 1871-2017 & 19 & Constituency & $1.37\times 10^{5}$ & $2.26\times 10^{4}$ & 5108\\ \hline
Ghana & 1992-2016 & 6 & Constituency & $3.75\times 10^{4}$ & $9.88\times 10^{3}$ & 1410\\ \hline
Hungary & 1990-2018 & 6 & Constituency & $5.32\times 10^{4}$ & $8.57\times 10^{3}$ & 936\\ \hline
India & 1951-2019 & 18 & Constituency & $5.69\times 10^{5}$ & $8.33\times 10^{4}$ & 8389\\ \hline
India & 2004-2019 & 4 & Polling Booth & $5.82\times 10^{2}$ & $1.89\times 10^{2}$ & 752786\\ \hline
Japan & 1947-2017 & 26 & Constituency & $2.88\times 10^{5}$ & $2.35\times 10^{4}$ & 4603\\ \hline
Kenya & 1961-2013 & 2 & Constituency & $3.72\times 10^{4}$ & $1.19\times 10^{4}$ & 417\\ \hline
Korea & 1948-2012 & 13 & Constituency & $6.17\times 10^{4}$ & $1.01\times 10^{4}$ & 2258\\ \hline
Lithuania & 1992-2020 & 8 & Constituency & $3.24\times 10^{4}$ & $3.98\times 10^{3}$ & 570\\ \hline
Malawi & 1994-2019 & 4 & Constituency & $2.31\times 10^{4}$ & $6.29\times 10^{3}$ & 755\\ \hline
Malaysia & 1959-2018 & 13 & Constituency & $3.41\times 10^{4}$ & $8.90\times 10^{3}$ & 2199\\ \hline
Myanmar & 2010-2015 & 2 & Constituency & $6.76\times 10^{4}$ & $2.32\times 10^{4}$ & 634\\ \hline
New Zealand & 1943-2020 & 9 & Constituency & $3.04\times 10^{4}$ & $6.94\times 10^{3}$ & 637\\ \hline
Nigeria & 2003-2019 & 2 & Constituency & $7.75\times 10^{4}$ & $2.20\times 10^{4}$ & 710\\ \hline
Pakistan & 1988-2013 & 3 & Constituency & $1.28\times 10^{5}$ & $2.45\times 10^{4}$ & 683\\ \hline
Papua New Guinea & 1972-2017 & 8 & Constituency & $5.07\times 10^{4}$ & $5.66\times 10^{3}$ & 841\\ \hline
Philippines & 1946-2013 & 17 & Constituency & $1.83\times 10^{5}$ & $2.63\times 10^{4}$ & 2525\\ \hline
Solomon Islands & 1967-2019 & 14 & Constituency & $3.67\times 10^{3}$ & $4.37\times 10^{2}$ & 543\\ \hline
Taiwan & 1986-2020 & 11 & Constituency & $2.33\times 10^{5}$ & $1.98\times 10^{4}$ & 482\\ \hline
Tanzania & 2005-2020 & 2 & Constituency & $5.37\times 10^{4}$ & $2.01\times 10^{4}$ & 492\\ \hline
Thailand & 1969-2011 & 12 & Constituency & $1.86\times 10^{5}$ & $1.46\times 10^{4}$ & 2263\\ \hline
Trinidad and Tobago & 1925-2020 & 13 & Constituency & $1.53\times 10^{4}$ & $5.12\times 10^{3}$ & 411\\ \hline
Uganda & 2006-2021 & 4 & Constituency & $4.45\times 10^{4}$ & $1.08\times 10^{4}$ & 1430\\ \hline
UK & 1832-2019 & 46 & Constituency & $3.43\times 10^{4}$ & $6.30\times 10^{3}$ & 23105\\ \hline
Ukraine & 1998-2019 & 5 & Constituency & $8.89\times 10^{4}$ & $1.67\times 10^{4}$ & 1072\\ \hline
United States & 1788-2020 & 167 & Congressional District & $1.14\times 10^{5}$ & $2.96\times 10^{4}$ & 33946\\ \hline
United States & 2000-2020 & 6 & County & $1.78\times 10^{5}$ & $2.00\times 10^{4}$ & 18905\\ \hline
Zimbabwe & 2005-2018 & 4 & Constituency & $1.77\times 10^{4}$ & $6.55\times 10^{3}$ & 743\\ \hline
\end{tabular}
}
\caption{Typical values of Margin and turnouts at different scales for different countries. The available data for the mentioned time spans were consolidated for each country and used to calculate the mean turnout and mean margin. The consolidated number of electoral units (in the last column) is calculated by adding the number of valid electoral units for all the elections that happened in the mentioned time span. The data for an electoral unit is considered to be valid if (a) a list of votes received by all the candidates is available, (b) at least two candidates are contesting, and (c) the turnout is non-zero. For example, in polling booth level data for India, lists of votes for all the polling booths are not always available. We could obtain valid data for $752786$ polling booths from the four elections held during the time span of $2004-2019$ for which the above-mentioned conditions were met. Only national-level elections are considered in this dataset.}
\label{tab:election_stats}
\end{table}

\newpage
\section{Dataset Validation and Analytical Framework}

This comprehensive dataset represents the empirical foundation upon which our discovery of electoral universality rests. The systematic collection, cleaning, and validation of data from 34 countries across six continents provides unprecedented analytical power for investigating fundamental patterns in democratic competition. Our dataset encompasses over 2.3 million electoral units when consolidated across all countries and scales, representing one of the most extensive cross-national electoral databases assembled for statistical physics analysis.

The quality-first approach in our data curation proves essential for uncovering robust statistical patterns. Rather than pursuing maximum geographical coverage, we prioritized statistical robustness by maintaining stringent quality thresholds for each included country. This methodology enabled us to demonstrate genuine universality in electoral statistics while providing sufficient statistical power to detect deviations indicative of electoral irregularities.

The multi-scale architecture of our dataset, particularly exemplified by the Indian election data spanning three orders of magnitude in electoral unit size, enables investigation of scale-invariant phenomena that would be impossible with single-scale analysis. The successful collection of polling booth level data for India and Canada, representing hundreds of thousands of electoral units, provides the granular resolution necessary for detecting universal statistical signatures across different scales of democratic aggregation.

With this empirically grounded foundation established, we possess the analytical infrastructure necessary to pursue our central research question: do universal statistical patterns emerge in democratic competition that transcend the specific institutional, cultural, and temporal contexts of individual electoral systems? The subsequent chapters will demonstrate how this carefully curated dataset enables the discovery of remarkable universality in electoral margins when properly scaled, revealing fundamental principles governing competitive democratic processes across vastly different contexts.
