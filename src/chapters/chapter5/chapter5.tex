\chapter{The Random Voting Model: When Chance Explains Choice}
\label{chap5}

In the previous chapter, we uncovered a remarkable universal pattern in electoral statistics: the scaled distribution of the specific margin (margin-to-turnout ratio) follows a consistent curve across diverse democracies, electoral systems, and scales. This universality emerged despite the vast differences in raw turnout and margin distributions across countries. We introduced the Random Voting Model (RVM) as a simple stochastic framework that remarkably captures this universal pattern with surprising accuracy.

Building on this discovery, this chapter provides a comprehensive analysis of the RVM from first principles. We will delve into its mathematical foundations, derive key analytical results, and demonstrate how this elegantly simple stochastic model can explain multiple empirical findings from electoral data. Through both analytical derivations and simulation results, we'll show that the RVM not only predicts the universal specific margin distribution but also connects turnout distributions to margin distributions across different electoral contexts.

\section{Introducing the Random Voting Model (RVM): Simplicity by Design}
\label{sec:RVM_intro}
The Random Voting Model is based on the premise that electoral outcomes can be understood through a minimal statistical framework that captures the essence of competition without modeling voter psychology or strategic behavior. The model is parameter-free beyond the turnout distribution and number of candidates, relying only on simple probabilistic principles.

In this Random Voting Model, $c_i$ number of candidates contest at $i$-th electoral unit with $n_i$ electors (voters) and each elector from the $i$-th electoral unit casts their vote for $j$-th candidate with a probability $p_{ij}$. These probabilities are assigned as follows: for each candidate, a number between $0$ and $1$ is drawn uniformly at random, which is assigned as an unnormalized probability weight $w_{ij}$ to that candidate. The weights are subsequently normalized to get the probability $p_{ij}, j = 1, 2 \dots c_i$ of receiving the vote of an elector. This can be mathematically stated as
\begin{equation}
    w_{ij} \sim \mathcal{U}(0, 1) \quad \text{and} \quad p_{ij} = \frac{w_{ij}}{\sum_k w_{ik}}, \text{ with } j = 1, 2 \dots c_i,
    \label{eq:RVM_prob}
\end{equation}
where $\mathcal{U}(0, 1)$ denotes a uniformly distributed random variable in $(0,1)$.

After each of the $n_i$ electors (voters) in $i$-th electoral unit casts their vote for some candidate $j$ independently with probability $p_{ij}$, the candidate receiving the most votes $V_{i, w}$ is declared the winner, and the candidate securing the next largest number of votes $V_{i, r}$ is the runner-up. The \emph{margin of victory} $M_i$ is then defined to be the vote difference between the winner and the runner-up: \emph{i.e.} $M_i = V_{i, w} - V_{i, r}$. The empirical election data we employ shows that the top three candidates, on average, account for nearly 87\% of all votes polled in an election.
Hence, as part of the model specification, we fix the number of candidates in each electoral unit to be three, i.e., $c_i = 3$ for all $i$.

The only input to this model is the raw turnout data, i.e., the number of voters (who actually voted) in each constituency. For the model simulation, we use the turnout data of real elections as the total number of voters in different constituencies. To understand how simulations are performed, consider this notional example: if a country has $N=100$ constituencies and data for five such elections is available. Then, the model is simulated on $500$ electoral units. The number of electors in each electoral unit is taken from the consolidated turnouts. Such a simulation of election is performed multiple times to get the average distributions.
\section{Analytical Derivation of the Universal Specific Margin Distribution}
\label{sec:RVM_first_symphony}
We now demonstrate how the RVM explains the universal distribution of scaled specific margin $Q_{\widetilde{\mu}}\left(\widetilde{\mu}\right)$ observed in the previous chapter. As mentioned in the previous section (\ref{sec:RVM_intro}), we consider the case where $3$ candidates are contesting in an election. The weight assigned for the $j$-th candidate of the $i$-th electoral unit is $w_{ij}$. These weights are drawn independently at random from a uniform distribution between $0$ and $1$.  The corresponding probability $p_{ij}$ of receiving votes is calculated by normalizing these weights. Hence, we have the following,
\begin{equation}
    w_{ij} \sim \mathcal{U}(0, 1) \text{ and } p_{ij} = \frac{w_{ij}}{\sum_{k=1}^3 w_{ik}}; \text{ with } j = 1, 2, 3.
\end{equation}
For the rest of the analysis in this chapter, we focus on a single ($i$-th) electoral unit with voter turnout $T$ and drop the corresponding index $i$ for brevity. Hence,
\mathtoolsset{centercolon}
\begin{equation}
    w_{ij} := w_j \text{ and } p_{ij} := p_j.
\end{equation}
\subsection{The Set-up: Large Turnout Limit}
For large turnout $(T \gg 1)$, it is reasonable to assume the number of votes received by $j$-th candidate is proportional to their probability $p_j$, in particular, $v_j \approx p_j T$. Hence, for $T \gg 1$, the \emph{margin} can be approximated as 
\begin{equation}
M \approx (p_{max} - p_{2nd \: max})T,
\end{equation}
where $p_{max}$ and $p_{2nd \:max}$ correspond to the largest and the second largest probabilities assigned to the candidates. For example, if the probabilities $p_1, p_2,$ and $p_3$ assigned to the 3 candidates are $0.1, 0.6,$ and $0.3$, then $p_{max} = p_2 = 0.6$ and $p_{2nd \:max} = p_3 = 0.3$. The margin $M$ can also be written in terms of $w_j$ as the following:
\begin{center}
\begin{align}
    \nonumber M &\approx \left(\frac{w_{max}}{w_1 + w_2 + w_3} - \frac{w_{2nd\:max}}{w_1 + w_2 + w_3}\right)T,\\
    \nonumber & = \left(\frac{w_{(3)}}{w_{(1)} + w_{(2)} + w_{(3)}} - \frac{w_{(2)}}{w_{(1)} + w_{(2)} + w_{(3)}}\right)T,\\
    & = \left(\frac{w_{(3)} - w_{(2)}}{w_{(1)} + w_{(2)} + w_{(3)}}\right)T,
\end{align}

\end{center}
where $w_{(k)}$ is the $k$-th order statistics \cite{BarBalNag2008}. Hence, 
\begin{center}
\begin{align}
    \frac{M}{T} \approx \frac{w_{(3)} - w_{(2)}}{w_{(1)} + w_{(2)} + w_{(3)}}.
    \label{eq:S6}
\end{align}
\end{center}
\subsection{Order Statistics: The Key to Understand Ranking}
Consider $n$ \emph{iid} random variables $\{X_1, X_2 \dots X_n\}$ drawn from a distribution $\rho(x)$. When arranged in ascending order, the random variable at the $k$-th spot is defined as the $k$-th order statistics. In particular, $n$-th and $1$-st order statistics correspond to the maximum and minimum of those $n$ random variables, respectively. The $k$-th order statistics of the random variable $X$ is denoted by $X_{(k)}$.

The joint probability density of all the order statistics of the above-mentioned $n$ random variables, $\mathbbm{P}\left(x_{(1)}, x_{(2)}, ... x_{(n)}\right)$, defined as the probability density that the random variable $X_{(k)}$ takes the value $x_{(k)}$ for $k \in \{ 1, 2, \dots, n\}$, is
\begin{equation}
    \mathbbm{P}\left(x_{(1)}, x_{(2)}, ... x_{(n)}\right) = n!\prod_{k=1}^{n}\rho\left(x_{(k)}\right).
\end{equation}
\subsection{Application to the Random Voting Model}
Now that we understand the joint probability density of the order statistics, for RVM, we have $n = 3$ and $\rho(x) = \mathcal{U}(0, 1)$. Hence we have,    
\begin{center}
    \begin{align}
        \mathbbm{P}\left(w_{(1)}, w_{(2)}, w_{(3)}\right) = 3! = 6; \text{ with } 0<w_{(1)}<w_{(2)}<w_{(3)}<1,
    \end{align}
\end{center}
and $\mathbbm{P}\left(w_{(1)}, w_{(2)}, w_{(3)}\right) = 0$ otherwise, with the following normalization:
\begin{equation}
    \int_{0}^{1}dw_{(3)}\int_{0}^{w_{(3)}}dw_{(2)}\int_{0}^{w_{(2)}} 6 dw_{(1)} = 1.
\end{equation}
From the joint probability distribution of all the order statistics, we calculate the approximate probability density function of specific margin $ M / T = \mu$ from Eq.~\eqref{eq:S6} as follows, 
\begin{center}
    \begin{align}
        \nonumber Q_{\mu}\left(\mu\right) & = 6 \nonumber \int_{0}^{1}dw_{(3)}\int_{0}^{w_{(3)}}dw_{(2)}\int_{0}^{w_{(2)}} \delta\left(\mu - \frac{w_{(3)}- w_{(2)}}{w_{(1)} + w_{(2)} + w_{(3)}}\right)dw_{(1)},\\
        & = 6 \int_{0}^{1}dw_{(3)}\int_{0}^{w_{(3)}} \frac{w_{(3)} - w_{(2)}}{\mu^2} \nonumber \mathbbm{1}_{0<\frac{w_{(3)} - \mu w_{(3)} - (1 + \mu)w_{(2)}}{\mu}<w_{(2)}} dw_{(2)},\\
        & = 6 \int_{0}^{1}dw_{(3)} \frac{(1 - \mu)(5 + 7\mu)w_{(3)}^2}{2(1 + \mu)^2(1 + 2\mu)^2}.\\
    \end{align}
\end{center}
Finally, after performing this integral, we get 
\begin{equation}
    Q_{\mu}\left(\mu\right) = \frac{(1 - \mu)(5 + 7\mu)}{(1 + \mu)^2(1 + 2\mu)^2}.
\end{equation}
The distribution $Q_{\mu}\left(\mu\right)$ does not depend on the turnout and is universal. Now, by a change of variable to scaled specific margin defined as $\widetilde{\mu} = \mu / \langle \mu \rangle$, we obtain its distribution $Q_{\widetilde{\mu}}\left(\widetilde{\mu}\right)$ to be
\begin{equation}
    Q_{\widetilde{\mu}}\left(\widetilde{\mu}\right) = \langle \mu \rangle ~ Q_{\mu}\left( \widetilde{\mu} \langle \mu \rangle \right) =  \frac{\langle \mu \rangle(1 - \widetilde{\mu} \langle \mu \rangle)(5 + 7\widetilde{\mu} \langle \mu \rangle)}{(1 + \widetilde{\mu} \langle \mu \rangle)^2(1 + 2\widetilde{\mu} \langle \mu \rangle)^2}, 
\end{equation}
where $\langle \mu\rangle = \frac{1}{2}+\ln \left(\frac{9 \sqrt[4]{3}}{16}\right)$.
This derived distribution $Q_{\widetilde{\mu}}\left(\widetilde{\mu}\right)$ is precisely the universal curve observed in the empirical data across 32 countries in the previous chapter. The remarkable agreement between theory and data as shown in Fig.~\ref{fig:RVM_mu} confirms that the RVM captures the essential statistical features underlying electoral competition.

\begin{figure}[H]
    \centering
    \includegraphics[width=0.95\textwidth]{chapters/chapter5/universality_empirical_analytical.pdf}
    \caption{$Q_{\widetilde{\mu}}\left(\widetilde{\mu}\right)$: The distribution is universal and does not depend on the turnout distribution. Each color indicates empirical $Q_{\widetilde{\mu}}\left(\widetilde{\mu}\right)$ for a specific country for which the election data is consolidated over several elections. The average of these empirical distributions (red open circles) closely follows the analytical curve (red line) and the averaged RVM predictions for each country (black open circles).}
    \label{fig:RVM_mu}
\end{figure}

\section{Predicting Margin Distributions from Turnout Distributions}
Having established that the RVM predicts the universal specific margin distribution, we now explore how the model can predict the margin distribution $Q_M(M)$ from an arbitrary turnout distribution $g(T)$. From the previous section, we have the distribution of the specific margin $\mu = M / T$ to be
\begin{equation}
    Q_{\mu}\left(\mu\right) = \frac{(1 - \mu)(5 + 7\mu)}{(1 + \mu)^2(1 + 2\mu)^2}.
\end{equation}
Through a simple change of variable $(M = \mu T)$ we get,
\begin{equation}
    \mathcal{P}(M|T) = \frac{(1 - M / T)(5 + 7M /T)}{T(1 + M / T)^2(1 + 2M / T)^2}.
\end{equation}
For an arbitrary turnout distribution $g(T)$, we obtain the distribution of $M$ to be,
\begin{equation}
    Q_M(M) = \int_{M}^{\infty}g(T)\mathcal{P}(M |T) dT = \int_{M}^{\infty}g(T)\frac{(1 - M / T)(5 + 7M /T)}{T(1 + M / T)^2(1 + 2M / T)^2} dT.
\end{equation}
Again with $u = T / M$, the above integral transforms to,
\begin{equation}
    Q_M(M) = \int_{1}^{\infty}g(Mu)\frac{u(u - 1)(5u + 7)}{(1 + u)^2 (2 + u)^2}du.
    \label{eq:pm}
\end{equation}
\subsection{Turnout Distribution Effects on Margin Distribution}
We compute $Q_M(M)$ for different turnout distributions $g(T)$. In particular, we take $g(T)$ to be (A) exponential, (B) power law, and (C) Gaussian distributions as they have vastly different tail behaviors. Finally we also consider a uniform turnout distribution with finite support.
\subsubsection{Exponential Turnout Distribution}
In this case $g(T) = \frac{1}{\tau}e^{-T / \tau}$, with $ \tau> 0$. Hence,
\begin{equation}
    Q_M(M) = \int_{1}^{\infty}\frac{1}{\tau}e^{-Mu / \tau} \frac{u(u - 1)(5u + 7)}{(1 + u)^2 (2 + u)^2}du,
\end{equation}
or,
\begin{equation}
    Q_M(M) = \frac{e^{-\frac{M}{\tau}}}{\tau^2} \left(4 e^{\frac{2 M}{\tau}} (\tau+M) \text{Ei}\left(-\frac{2 M}{\tau}\right)-9 e^{\frac{3 M}{\tau}} (\tau+2 M) \text{Ei}\left(-\frac{3 M}{\tau}\right)-4 \tau\right), 
\end{equation}
where $\text{Ei}(x) = \int_{-\infty}^{x}\frac{e^t}{t}dt$. At large margin limit $(M \rightarrow \infty)$, the asymptotic behavior of the distribution is the following (up to the leading order of $M$):
\begin{equation}
    Q_M(M)= \frac{\tau}{3M^2}e^{-M/\tau}.
\end{equation}
This suggests that in the large margin limit, both the margin and its corresponding turnout distribution have an exponential decay with the same rate.
\subsubsection{Power law Turnout Distribution}
In this case $g(T) = \frac{\alpha - 1}{T_{min} ^{1 -\alpha}} T ^ {-\alpha}$, with $\alpha > 1$ and $T>T_{min}$. Hence we have,
\begin{equation}
    Q_M(M) = \int_{1}^{\infty}\frac{\alpha - 1}{T_{min} ^{1 -\alpha}} (Mu) ^ {-\alpha} \frac{u(u - 1)(5u + 7)}{(1 + u)^2 (2 + u)^2}du,
\end{equation}
or,
\begin{equation}
    Q_M(M) = C(M)\frac{\alpha - 1}{T_{min} ^{1 -\alpha}} (M) ^ {-\alpha}, 
    \label{eq:powerlaw}
\end{equation}
where,
\begin{numcases}{C(M) = }
    I_1(\infty) - I_1(T_{min} / M) , \text{if } M\leq T_{min}\\
    I_1(\infty) - I_1(1), \text{otherwise,}
\end{numcases}
with,
\begin{equation}
    I_1(y) = \int \frac{y^{1 - \alpha}(y - 1)(5y + 7)}{(1 + y)^2 (2 + y)^2}dy,
\end{equation}
and,
\begin{numcases}{I_1(y)=}
     -\frac{4}{y+1}+\frac{9}{2 (y+2)}-\frac{1}{4} 7 \ln (y)+4 \ln (y+1)-\frac{9}{4} \ln (y+2), \text{if } \alpha = 2 \\
     \frac{y^{2-\alpha } \left(16 \, _2F_1(2,2-\alpha ;3-\alpha ;-y)-9 \, _2F_1\left(2,2-\alpha ;3-\alpha ;-\frac{y}{2}\right)\right)}{4 (\alpha -2)}, \text{otherwise,} \\
\end{numcases}
where ${}_{2}F_{1}(a,b;c;z)$ is a hypergeometric function \cite{abramowitz_stegun}, defined as,
\begin{align*}
    {\displaystyle {}_{2}F_{1}(a,b;c;z)  =\sum _{n=0}^{\infty }{\frac {(a)_{n}(b)_{n}}{(c)_{n}}}{\frac {z^{n}}{n!}}=1+{\frac {ab}{c}}{\frac {z}{1!}}+{\frac {a(a+1)b(b+1)}{c(c+1)}}{\frac {z^{2}}{2!}}+\cdots .}\\
\end{align*}

It is evident from Eq.~\eqref{eq:powerlaw} that for $M > T_{min}$, the margin distribution decays with a power law exponent $\alpha$, exactly the same as the turnout distribution.
\subsubsection{Gaussian Turnout Distribution}
In this case $g(T) = C_0 e^{-(T/T_0)^2}$, with $T>0$. Hence,
\begin{equation}
     Q_M(M) = \int_{1}^{\infty} C_0 e^{-(Mu/T_0)^2}\frac{u(u - 1)(5u + 7)}{(1 + u)^2 (2 + u)^2}du.
\end{equation}
At large margin limit $(M \rightarrow \infty)$, the asymptotic behavior of the distribution is the following (up to the leading order of $M$):
\begin{equation}
    Q_M(M) = \frac{C_0}{12}\left(\frac{T_0}{M}\right)^4 e^{-\left(M/ T_0\right)^2}, 
\end{equation}
and it has a Gaussian decay similar to the corresponding turnout distribution.\\

From the asymptotic analysis of the margin distributions for the three above-mentioned turnout distributions, we provide strong evidence that the tails of the margin distributions mimic that of the corresponding turnout distribution. For completeness, we also compute the margin distribution corresponding to a uniform turnout distribution which has a finite support (no tail behavior).
\subsubsection{Uniform Turnout Distribution}
In this case $g(T) = \frac{1}{b - a}$, when $T \in [a, b]$, otherwise  $g(T) = 0$. Hence,
\begin{numcases}{Q_M(M)= }
    \frac{1}{b - a}\int_{a/M}^{b/M}\frac{u(u - 1)(5u + 7)}{(1 + u)^2 (2 + u)^2}du , \text{if } M\leq a\\
    \frac{1}{b - a}\int_{1}^{b/M}\frac{u(u - 1)(5u + 7)}{(1 + u)^2 (2 + u)^2}du, \text{otherwise,}
\end{numcases}
or, 
\begin{numcases}{Q_M(M)= }
    \frac{1}{b - a} \left(I_2(b / M) - I_2(a / M)\right), \text{if } M\leq a\\
    \frac{1}{b - a}\left(I_2(b / M) - I_2(1)\right),  \text{if } a > M \geq b\\
    0,  \text{ otherwise,}
\end{numcases}
where, 
\begin{equation}
    I_2(y) = \int \frac{y(y - 1)(5y + 7)}{(1 + y)^2 (2 + y)^2}dy = -\frac{4}{y+1}+\frac{18}{y+2}-4 \ln (y+1)+9 \ln (y+2).
\end{equation}
\subsection{RVM Simulations with Synthetic Turnout Distributions}
We perform simulation using the following four different synthetic turnout distributions, to further validate our analytical results, specifically the tail behavior of the margin distribution.
\begin{enumerate}
    \item \textbf{Gaussian Turnout Distribution:} $g(T) = \frac{1}{\sigma\sqrt{2\pi}}\exp\left(-\frac{(T - \mu)^2}{2\sigma^2}\right), \text{ with } \mu = 50000$, $\sigma = 10000$ and $T > 0$.
    \item \textbf{Exponential Turnout Distribution:} $g(T) = \frac{1}{\tau} \exp{\left(-\frac{T}{\tau}\right)}, \text{ with } \tau = 50000$.
    \item \textbf{Power law Turnout Distribution:} $g(T) = \frac{\alpha - 1}{T_{min} ^{1 -\alpha}} T ^ {-\alpha}$, with $\alpha = 2$ and $T_{min} = 100$ (minimum possible turnout).
    \item \textbf{Uniform Turnout Distribution:} $T \sim \mathcal{U} (a, b)$, with $a = 100$ and $b = 100000$. $\mathcal{U}(a, b)$ denotes uniform distribution between the range $a$ and $b$.
\end{enumerate}

Each of the RVM simulations was performed on $10^6$ electoral units, with turnouts (rounded down to the nearest integer) drawn from one of these four distributions. The simulation demonstrates that the tail of the margin distribution mimics the turnout distribution's tail.  This is evident in Fig.~\ref{fig:margin_turnout_dists}(a), (b), and (c). The tail of the margin distribution (Fig.~\ref{fig:margin_turnout_dists} (c)) corresponding to power law turnouts decays with the same power law exponent.  In the simulation with Gaussian turnout distribution, we find the tail of the margin distribution also has a Gaussian falloff (Fig.~\ref{fig:margin_turnout_dists} (a)). Similarly, the margin distribution corresponding to exponential turnouts has an exponential tail (Fig.~\ref{fig:margin_turnout_dists} (b)). As the probability density function of uniform turnout distribution and corresponding margin distribution have finite supports, their tails can not be properly defined. We find a sharp cutoff in the corresponding margin distribution. The analytical (semi-analytical for Gaussian turnout) predictions for the margin distributions (shown as black lines in Fig.~\ref{fig:margin_turnout_dists}) corresponding to all four aforementioned turnout distributions are in excellent agreement with the RVM simulation. In empirical county-level election data of the United States, the heavy-tailed decay of the turnout distribution is reflected in the corresponding margin distribution (Fig.~\ref{fig:margin_turnout_dists}(e)). In Fig.~\ref{fig:margin_turnout_dists}(f), we see a similar decay trend in both margin and turnout distribution, which correspond to congressional district-level election data of the USA. 

\begin{figure}[H]
    \centering
    \includegraphics[width=\textwidth]{chapters/chapter5/margin_turnout_dists.pdf}
    \caption{The margin distribution $Q_M(M)$ is plotted with the corresponding turnout distribution $g(T)$ to demonstrate that the tails of both these distributions are correlated. Panels (a), (b), (c), and (d) correspond to Gaussian, exponential, power law, and uniform turnout distributions, respectively. Blue open circles denote the turnout distributions. Red open circles denote the margin distribution computed through RVM simulations. Black solid lines correspond to the margin distribution computed using Eq.~\ref{eq:pm}. For exponential, power law, and uniform turnout distributions, the integration was analytically calculated, and for Gaussian turnout distribution, it was evaluated numerically. Panels (e) and (f) depict the margin and turnout distribution for the county-level and congressional district-level election data of the USA, respectively.}
    \label{fig:margin_turnout_dists}
\end{figure}
\subsection{The Necessity of Scaling in Margin Distribution Analysis}
While the RVM successfully explains the shape of the margin distributions $Q_M(M)$ and capture their tail behaviors remarkably well, it fails to predict the empirical mean of those distributions. To overcome this limitation, we scale the margin distribution by its mean $\langle M \rangle$, and investigate the predictive power of the RVM on the scaled margin distribution $Q_{\widetilde{M}}(\widetilde{M})$.
\subsubsection{Predicting Scaled Margin Distributions Across Countries}
As demonstrated in the previous chapter, different countries exhibit vastly different turnout distributions $g(T)$ that reflect distinct electoral systems and contexts. The key question is whether the RVM can capture the resulting variations in scaled margin distributions $Q_{\widetilde{M}}(\widetilde{M})$.

Figure \ref{fig:RVM_country_predictions} (b-g) shows that the RVM predictions (solid lines) achieve excellent agreement with empirical scaled margin distributions (open circles) across diverse countries including India, USA, South Korea, Canada, Japan, and Germany. Notably, the model faithfully captures the disparate decay features observed in different countries — from the sharp cutoff characteristic of German elections to the heavier tails seen in Indian and Japanese data. This demonstrates that raw turnout data carries intrinsic information about the margin distribution, which the RVM effectively leverages without requiring any parameter tuning.

\begin{figure}[H]
    \centering
    \includegraphics[width=\textwidth]{chapters/chapter5/turnout_margin_empirical_simulation_distribution_pc.pdf}
    \caption{RVM predictions of scaled margin distributions across diverse electoral systems. (a) Turnout distributions $g(T)$ showing the diversity in electoral contexts. (b-g) Empirical scaled margin distributions $Q_{\widetilde{M}}(\widetilde{M})$ (open circles) compared with parameter-free RVM predictions (solid lines) for India, USA, South Korea, Canada, Japan, and Germany. The shaded regions represent prediction variability from multiple RVM realizations.}
    \label{fig:RVM_country_predictions}
\end{figure}

\subsubsection{Scale-Independent Predictive Accuracy}
The predictive power of the RVM extends across different electoral scales without requiring parameter adjustments. As shown in Figure \ref{fig:RVM_scale_predictions} (b-g), the same model accurately predicts scaled margin distributions at both constituency-level (larger scale) and polling booth/county-level (smaller scale) for India, USA, and Canada. Despite turnout distributions that differ by orders of magnitude in scale and exhibit completely different shapes, the RVM maintains its predictive accuracy, confirming that the underlying statistical principles captured by the model are truly scale-invariant.

\begin{figure}[H]
    \centering
    \includegraphics[width=\textwidth]{chapters/chapter5/turnout_margin_empirical_simulation_distribution_diff_scale.pdf}
    \caption{Scale-independent RVM predictions. (a) Turnout distributions $g(T)$ at two different electoral scales: dashed lines for smaller scales (polling booths/counties), solid lines for larger scales (constituencies/congressional districts). (b-g) Empirical $Q_{\widetilde{M}}(\widetilde{M})$ (open circles) and RVM predictions (lines) demonstrate excellent agreement across both scales despite vastly different turnout characteristics.}
    \label{fig:RVM_scale_predictions}
\end{figure}

\subsubsection{Comprehensive Validation Across All Countries}
Figure \ref{fig:RVM_all_countries} provides a comprehensive demonstration of the RVM's predictive accuracy across our complete dataset of 32 countries. The remarkable agreement between empirical distributions (colored circles) and parameter-free RVM predictions (black lines) across such diverse electoral systems — spanning different continents, political cultures, and institutional arrangements — underscores the model's robustness and the fundamental nature of the statistical principles it captures.

\begin{figure}[H]
    \includegraphics[width=\textwidth]{chapters/chapter5/scale_margin_distribution_32_countries.pdf}
    \caption{Comprehensive validation of RVM predictive accuracy. Empirical scaled margin distributions $Q_{\widetilde{M}}(\widetilde{M})$ (colored circles) compared with parameter-free RVM predictions (black solid lines) across all 32 countries in our dataset, demonstrating consistent predictive accuracy across diverse electoral systems.}
    \label{fig:RVM_all_countries}
\end{figure}

\section{Conclusion: The Power and Limitations of RVM}

In this chapter, we have provided a comprehensive analysis of the Random Voting Model (RVM) from its first principles. We demonstrated how this elegantly simple model generates the universal distribution of scaled specific margins observed across diverse electoral systems worldwide. Through rigorous mathematical derivations, we showed that the model successfully predicts this universality without requiring any parameters beyond the turnout distribution and number of candidates.

Furthermore, we established that the RVM provides powerful insights into how turnout distributions shape margin distributions across different electoral contexts. Through both analytical solutions and simulations with various synthetic turnout distributions (Gaussian, exponential, power law, and uniform), we demonstrated that margin distributions inherit the tail behavior of their corresponding turnout distributions. This finding explains why electoral margin distributions can vary dramatically across countries while still adhering to the universal specific margin pattern.

The model's ability to predict scaled margin distributions from raw turnout data alone—without any adjustments or free parameters—is remarkable and represents a significant advance in our understanding of electoral statistics. It suggests that beneath the apparent complexity and diversity of electoral systems worldwide lies a fundamental statistical principle that governs competitive selection processes.

However, the RVM also has limitations. While it excellently predicts the shape and tail behavior of margin distributions, it struggles to reliably predict the mean margin. This necessitates scaling by empirical means when comparing model predictions to real-world data. Additionally, the model's simplified assumption of random voting does not account for strategic voting behavior, partisan affiliations, or other psychological and sociological factors that influence real elections.

Despite these limitations, the success of this minimal statistical model in capturing key universal features of electoral competition demonstrates the power of stochastic approaches in understanding complex social phenomena. It reveals that certain macroscopic patterns in electoral systems may be governed more by basic statistical principles than by the intricacies of voter psychology or electoral rules.

In the next chapter, we will build upon this foundation to explore prediction of several other key electoral statistics.